%%%%%%%%%%%%%%%%%%
% Document class %
%%%%%%%%%%%%%%%%%%

\documentclass[a4paper, 12pt]{article}


%%%%%%%%%%%%
% Packages %
%%%%%%%%%%%%

\usepackage[english]{babel}
\usepackage[noheader]{packages/sleek}
\usepackage{packages/sleek-title}

\usepackage{tabularx}
\usepackage{colortbl}


%%%%%%%%%%%%%%%%%%%%
% Title page setup %
%%%%%%%%%%%%%%%%%%%%

\logo{resources/pdf/logo-uliege.pdf}
\institute{University of Liège}
\faculty{Faculty of Applied Science}
\title{Securing network with firewalls and\\NATs}
\subtitle{Step 2: high-level rules}
\author{Maxime \textsc{Meurisse}\\Valentin \textsc{Vermeylen}}
\context{Master in Civil Engineering}
\date{Academic year 2020-2021}


%%%%%%%%%%
% Others %
%%%%%%%%%%

\definecolor{lightgray}{rgb}{0.83, 0.83, 0.83}


%%%%%%%%%%%%
% Document %
%%%%%%%%%%%%

\begin{document}
	\maketitle
	
	\section{High-level rules}
	
	The rules for each firewall are defined in Tables \ref{tab:rules.fw1}, \ref{tab:rules.fw2}, \ref{tab:rules.fw3} and \ref{tab:rules.fw4}. For each zone, from more secured to less secured, we define first the rules for incoming traffic and then the rules for outgoing traffic. We end with a rule that deny all and log.
	
	\subsection{Firewall 1}
	
	\begin{table}[H]
	    \centering
	    \begin{tabularx}{\textwidth}{|l||c|c||c|c||c|c|X|}
	        \hline
	        \textbf{\#} & \textbf{Source} & \textbf{Port} & \textbf{Destination} & \textbf{Port} & \textbf{Protocol} & \textbf{Action} & \textbf{Comments}\\ \hline
	        \hline
	        \rowcolor{lightgray}
	        \multicolumn{8}{|c|}{Incoming traffic \emph{z\_servers\_1}}\\ \hline
	        1 & * & * & * & * & * & * & *\\ \hline
	        \rowcolor{lightgray}
	        \multicolumn{8}{|c|}{Outgoing traffic \emph{z\_servers\_1}}\\ \hline
	        2 & * & * & * & * & * & * & *\\ \hline
	        \rowcolor{lightgray}
	        \multicolumn{8}{|c|}{Incoming traffic \emph{z\_public}}\\ \hline
	        1 & * & * & * & * & * & * & *\\ \hline
	        \rowcolor{lightgray}
	        \multicolumn{8}{|c|}{Outgoing traffic \emph{z\_public}}\\ \hline
	        2 & * & * & * & * & * & * & *\\ \hline
	        \rowcolor{lightgray}
	        \multicolumn{8}{|c|}{Other}\\ \hline
	        3 & * & * & * & * & * & deny, log & Should not happen. Log to be sure.\\ \hline
	    \end{tabularx}
	    \caption{Rules for firewall \emph{FW1}.}
	    \label{tab:rules.fw1}
	\end{table}
	
	\subsection{Firewall 2}
	
	\begin{table}[H]
	    \centering
	    \begin{tabularx}{\textwidth}{|l||c|c||c|c||c|c|X|}
	        \hline
	        \textbf{\#} & \textbf{Source} & \textbf{Port} & \textbf{Destination} & \textbf{Port} & \textbf{Protocol} & \textbf{Action} & \textbf{Comments}\\ \hline
	        \hline
	        \rowcolor{lightgray}
	        \multicolumn{8}{|c|}{Incoming traffic \emph{z\_subnet\_2}}\\ \hline
	        1 & * & * & * & * & * & * & *\\ \hline
	        \rowcolor{lightgray}
	        \multicolumn{8}{|c|}{Outgoing traffic \emph{z\_subnet\_2}}\\ \hline
	        2 & * & * & * & * & * & * & *\\ \hline
	        \rowcolor{lightgray}
	        \multicolumn{8}{|c|}{Incoming traffic \emph{z\_lweb}}\\ \hline
	        1 & * & * & * & * & * & * & *\\ \hline
	        \rowcolor{lightgray}
	        \multicolumn{8}{|c|}{Outgoing traffic \emph{z\_lweb}}\\ \hline
	        2 & * & * & * & * & * & * & *\\ \hline
	        \rowcolor{lightgray}
	        \multicolumn{8}{|c|}{Other}\\ \hline
	        3 & * & * & * & * & * & deny, log & Should not happen. Log to be sure.\\ \hline
	    \end{tabularx}
	    \caption{Rules for firewall \emph{FW2}.}
	    \label{tab:rules.fw2}
	\end{table}
	
	\subsection{Firewall 3}
	
	\begin{table}[H]
	    \centering
	    \begin{tabularx}{\textwidth}{|l||c|c||c|c||c|c|X|}
	        \hline
	        \textbf{\#} & \textbf{Source} & \textbf{Port} & \textbf{Destination} & \textbf{Port} & \textbf{Protocol} & \textbf{Action} & \textbf{Comments}\\ \hline
	        \hline
	        \rowcolor{lightgray}
	        \multicolumn{8}{|c|}{Incoming traffic \emph{z\_nfs}}\\ \hline
	        1 & * & * & * & * & * & * & *\\ \hline
	        \rowcolor{lightgray}
	        \multicolumn{8}{|c|}{Outgoing traffic \emph{z\_nfs}}\\ \hline
	        2 & * & * & * & * & * & * & *\\ \hline
	        \rowcolor{lightgray}
	        \multicolumn{8}{|c|}{Incoming traffic \emph{z\_u3}}\\ \hline
	        1 & * & * & * & * & * & * & *\\ \hline
	        \rowcolor{lightgray}
	        \multicolumn{8}{|c|}{Outgoing traffic \emph{z\_u3}}\\ \hline
	        2 & * & * & * & * & * & * & *\\ \hline
	        \rowcolor{lightgray}
	        \multicolumn{8}{|c|}{Other}\\ \hline
	        3 & * & * & * & * & * & deny, log & Should not happen. Log to be sure.\\ \hline
	    \end{tabularx}
	    \caption{Rules for firewall \emph{FW3}.}
	    \label{tab:rules.fw3}
	\end{table}
	
	\subsection{Firewall 4}
	
	\begin{table}[H]
	    \centering
	    \begin{tabularx}{\textwidth}{|l||c|c||c|c||c|c|X|}
	        \hline
	        \textbf{\#} & \textbf{Source} & \textbf{Port} & \textbf{Destination} & \textbf{Port} & \textbf{Protocol} & \textbf{Action} & \textbf{Comments}\\ \hline
	        \hline
	        \rowcolor{lightgray}
	        \multicolumn{8}{|c|}{Incoming traffic \emph{z\_subnet\_1}}\\ \hline
	        1 & * & * & * & * & * & * & *\\ \hline
	        \rowcolor{lightgray}
	        \multicolumn{8}{|c|}{Outgoing traffic \emph{z\_subnet\_1}}\\ \hline
	        2 & * & * & * & * & * & * & *\\ \hline
	        \rowcolor{lightgray}
	        \multicolumn{8}{|c|}{Incoming traffic \emph{z\_servers\_2}}\\ \hline
	        1 & * & * & * & * & * & * & *\\ \hline
	        \rowcolor{lightgray}
	        \multicolumn{8}{|c|}{Outgoing traffic \emph{z\_servers\_2}}\\ \hline
	        2 & * & * & * & * & * & * & *\\ \hline
	        \rowcolor{lightgray}
	        \multicolumn{8}{|c|}{Other}\\ \hline
	        3 & * & * & * & * & * & deny, log & Should not happen. Log to be sure.\\ \hline
	    \end{tabularx}
	    \caption{Rules for firewall \emph{FW4}.}
	    \label{tab:rules.fw4}
	\end{table}
	
	\section{Explanations}
	
	to do
\end{document}
